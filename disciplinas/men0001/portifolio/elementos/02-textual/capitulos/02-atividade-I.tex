\chapter{Análise da Transcrição - Explicando o Funcionamento do Micro-ondas}
\label{chap: transc}

% Nesta atividade, analisou-se a(s) interação(s) professor-aluno(s); aluno(s)-professor e aluno-aluno(s), bem como ocorreram estas interações do ponto de vista dos discursos entre os atores envolvidos, para tanto, utilizou-se como recurso a ferramenta dos autores \cite{MORTIMER:2002} em uma transcrição de uma aula sobre o funcionamento do forno micro-ondas \emph{(Tentando Explicar o Funcionamento do Forno Micro-Ondas)}. A transcrição é dividida em 02 duas cenas:
% \begin{itemize}
%     \item cena-1A: \emph{Tentando explicar o funcionamento do forno de micro-ondas}; e
%     \item cena-1B: \emph{Comparando o funcionamento dos fornos convencionais e de micro-ondas}.    
% \end{itemize}
% A metodologia utilizada para esta análise, consiste na caracterização, classificação e qualificação dos discursos apresentados na transcrição, segundo a abordagem sugerida pelos autores. Esta atividade foi desenvolvida ao longo de três aulas que ocorreram entre os dias 19/04 e 28/04, envolvendo:
% \begin{enumerate}[label=\alph *)]
%     \item Leitura completa do artigo e elaboração de questões para discussão em sala de aula - \emph{19/04};
% \end{enumerate}


\section{Habilidades do Professor}
\label{subsec: transcHabProf}

\subsection{Criando um ambiente construtivo}
A atividade de investigação é proposta em uma aula anterior à transcrição de que trata esta análise, e consiste basicamente na resolução das seguintes questões:
\begin{enumerate}[label=\roman *)]
    \item Relacione abaixo os aparelhos e/ou dispositivos que podem ser usados em uma casa, destinados a provocar aquecimento;
    \item Agrupe-os e explique como eles funcionam.
\end{enumerate}
Por estas questões, identifica-se já de antemão uma certa aproximação da proposta do professor com a abordagem didática do \emph{Ensino por Investigação}. Na perspectiva de \citeonline{SASSERON:2015} o Ensino por Investigação tem por objetivos, \emph{``levar os estudantes a realizarem investigação e de desenvolver entre os estudantes um entendimento sobre o que seja a investigação científica''}. Uma característica desta abordagem, reside em promover o papel ativo do aluno na construção do entendimento sobre os conhecimentos científicos, esta ação relaciona-se com a(s) intenção(ões) do professor de oportunizar situações que desenvolvam o protagonismo em sala de aula por parte dos estudantes, uma vez que é transferida a eles a responsabilidade de relacionar, classificar, agrupar e por fim explicar o funcionamento de cada aparelho encontrado em seu ambiente domiciliar. \cite{MORTIMER:2002} sintetiza as intenções do professor em seis categorias inter-relacionadas, tendo uma delas por foco \emph{engajar os estudantes, intelectualmente, no desenvolvimento inicial da "estória científica"}. Neste sentido, estas pesquisas concordam com o papel do professor como o
\begin{citacao}
    ``\ldots de fazer com que a turma se engaje com as discussões e, ao mesmo tempo em que travam contato com fenômenos naturais, pela busca de resolução de um problema, exercitam práticas e raciocínios de comparação, análise e avaliação bastante utilizadas na prática científica.'' \cite[p. ~58]{SASSERON:2015}
\end{citacao}

Partindo-se deste pressuposto e lendo na transcrição as falas dos alunos, notou-se que grande parte das investigações giraram em torno da natureza de funcionamento do forno de micro-ondas, onde os alunos são convidados indiretamente a  obter uma melhor compreensão dos conceitos físicos envolvidos na questão. Inicialmente o fazem por meio do diálogo entre os pares, o trecho a seguir ilustra um pouco desta dinâmica:
\begin{itemize}
    \item 2. [J]: A gente coloca\ldots da radiação\ldots como a gente faz? É irradiação ou radiação?
    \item 3. [E]: Eu acho que é i
    \item 4. [F]: Vou procurar no dicionário\ldots [pega na sua mala um dicionário em edição de bolso]    
\end{itemize}
Se tratando de conceitos científicos já consolidados e estruturados, é fundamental para o estudante estar diante do termo preciso atribuído ao conceito. Pela transcrição não é possível afirmar se o(a) aluno(a) [J] o faz despretensiosamente ao buscar num primeiro momento, delimitar o termo correto antes mesmo de apropriar-se do conceito, mas ao estabelecê-lo, consolida um dos três eixos estruturantes da \emph{Alfabetização Científica} apresentados na pesquisa, sendo ele
\begin{citacao}
    ``\ldots a compreensão básica de termos e conceitos científicos, retratando a importância de que os conteúdos curriculares próprios das ciências sejam debatidos na perspectiva de possibilitar o entendimento conceitual;'' \opcit[p. ~57]{SASSERON:2015}
\end{citacao}
Após lerem a definição da palavra \emph{irradiação} encontrada no dicionário, os alunos começam a se questionar,
\begin{itemize}
    \item 6. [F]: “ato ou efeito de irradiar, bombardeio de uma substância por um feixe de partículas”
    \item 7. [J]: Mas\ldots até aí\ldots 
    \item 8. [E]: Então vê radiação\ldots
\end{itemize}
Não satisfeitos com a definição da primeira palavra encontrada no dicionário, buscam pela definição da segunda: \emph{radiação} 
\begin{itemize}
    \item 11. [J]: Como assim, ele provoca luz?\ldots mas o aquecimento, o calor\ldots de onde vem o calor? Da luz?
    \item 12. [E]: Do feixe de luz
    \item 13. [J]: Será\ldots assim? Eu não sei\ldots 
    \item 14. [E]: Eu acho que também deve ser um tipo de filamento
    \item 15. [F]: Achei\ldots “radiar: emitir ondas e energia calorífica, luminosa, etc. Cintilar, resplandecer”.
\end{itemize}
Deste trecho em diante, percebe-se um movimento integrado na direção de constituir as bases norteadoras das discussões e os fundamentos das construções argumentativas que se estenderão ao longo da atividade. Não é evidente, mas pode-se inferir com certa cautela, que o aluno [J] procura encontrar nas definições do dicionário substantivos como: \emph{filamento}; \emph{calor} e \emph{resistor} cujo os quais estão relacionados ao aquecimento, mas que de alguma forma lhe é mais familiar.
Esta etapa é delicada e exige destreza do professor em identificar sob quais bases os estudantes irão fundamentar seus posicionamentos diante da proposta, além disso, ter clareza sobre os objetivos da atividade é essencial tanto para os alunos, quanto para o professor.

É claro que os significados das palavras \emph{radiação} e \emph{irradiação} assim encontradas no dicionário, contribuem muito pouco para elucidar os conceitos físicos por trás da tecnologia de aquecimento utilizada em micro-ondas, pelo contrário, arrisco a dizer que podem até confundir os alunos\footnote{Radiação e Irradiação em Física, diz respeito ao transporte de energia térmica na forma de calor, enquanto um está relacionado ao transporte em si, o outro está relacionado às circunstâncias em que ocorre este transporte. O aquecimento no forno de micro-ondas se dá em termos do trabalho termodinâmico e não do calor.}, mas se essa conduta for bem explorada sob a supervisão do professor, pode tornar os resultados didaticamente mais interessantes, como indicado na pesquisa
\begin{citacao}
    ``\ldots o ensino por investigação exige que o professor valorize pequenas ações
    do trabalho e compreenda a importância de colocá-las em destaque como, por
    exemplo, os pequenos erros e/ou imprecisões manifestados pelos estudantes, as
    hipóteses originadas em conhecimentos anteriores e na experiência de sua turma,
    as relações em desenvolvimento.'' \cite[p. ~58]{SASSERON:2015}
\end{citacao}
Vemos então que a atividade em análise permitiu a iniciação destes processos e não custou muito para que os estudantes consultassem a orientação do professor, como veremos logo em sequência
\begin{itemize}
    \item 27. [J]: Eu vou ler de novo\ldots o que diz sobre radiação pra gente pensar o que é\ldots. é o aquecimento através de ondas, não é? Por isso é que chama micro-ondas\ldots a luz eu sei que tem a luz\ldots a onda é aquela tal\ldots mas\ldots quando a gente colocar um prato não aquece por igual\ldots as vezes uma parte fica fria e a outra\ldots
    \item 29. [J]: É?\ldots As ondas têm irregularidades?
    \item 30. [E]: Chama o professor\ldots mostra pra ele\ldots
    \item 31. [J]: Professor tá difícil\ldots essa coisa de micro-ondas\ldots
\end{itemize}
A resposta do professor revela uma segunda intenção
\begin{itemize}
    \item 32. [Pr]: Deixa eu ver\ldots ajudar na discussão\ldots Eu quero cozinhar uma carne, por exemplo, pra isso eu posso dispor do fogão\ldots a combustível\ldots a gás e de um forno de micro-ondas\ldots a primeira coisa\ldots é o tempo de cozimento eles são iguais?
\end{itemize}
Esta pergunta, de acordo com \cite{MORTIMER:2002}, tem por intenção expor mais a visão e o entendimento dos estudantes a fim de explorá-las e possibilitar a reflexão/articulação de suas próprias ideias. Usa a \emph{Abordagem Comunicativa} apresentada em \Ibidem[p. ~287]{MORTIMER:2002}, a qual é caracterizada pelos diferentes padrões de iterações entre professor-aluno, aluno-aluno e vice-versa. Segundo este viés as dimensões extremas da abordagem como o \emph{discurso dialógico ou de autoridade} e o \emph{discurso interativo ou não-interativo}, são combinados entre si formando quatro classes de interação, isto ocorreu quando professor e aluno(s) consideraram vários pontos de vista \emph{(discurso interativo dialógico)}.
\begin{itemize}
    \item 33. [J]: Não\ldots o micro-ondas a gente pode controlar a intensidade e o tempo\ldots
    \item 34. [Pr]: Ótimo\ldots o micro-ondas doura as coisas?
\end{itemize}   
A presença da palavra \emph{"Ótimo"} na fala do professor, reforça a participação dos alunos, ainda que os argumentos como intensidade e controle do tempo de cozimento não sejam características inerentes apenas ao forno de micro-ondas, todavia é observado este encorajamento quando é recomendado a organizarem suas falas um por vez.
\begin{itemize}
    \item 36. [Pr]: Pera aí\ldots um por vez\ldots
    \item 37. [F]: Não\ldots porque ele cozinha por dentro\ldots
    \item 38. [Pr]: Ótimo vocês já tão começando a levantar hipóteses de que um processo diferente tá ocorrendo\ldots
    \item 39. [E]: Aqui a gente tinha feito\ldots que ele aquece substâncias que tinham 50\% de água\ldots
    \item 40. [Pr]: Por que 50\% de água?
    \item 41. [E]: Não só 50\% de água\ldots 50\% ou mais\ldots e que o prato não aquece por que\ldots não absorve as ondas\ldots
    \item 42. [J]: Eu não entendo esse negócio\ldots de ondas\ldots ainda não entrou na minha cabeça.
    \item 43. [Pr]: Tá bom então coloque uma interrogação nisto\ldots
\end{itemize}
Outra classe de iteração \emph{(iterativo dialógico de autoridade)} é vista aqui ao final quando o professor recomenda a [J] identificar os pontos não compreendidos dos conceitos envolvidos.

Visto que houve interações entre professor, estudantes e o objeto de investigação, além do engajamento dos estudantes entre si e com a proposta trazida pelo professor, qualifica-se que nesta atividade o professor foi capaz de criar um ambiente construtivo propício ao processo de aprendizagem nos termos da Alfabetização Científica conforme indicado na literatura e como destacado, foi possível identificar algumas de suas intenções a partir da elaboração da proposta bem como de seus questionamentos, observou-se ainda a elaboração das bases argumentativas que permearão o decorrer da atividade. Na sequência classificaremos os diversos aspectos discursivos utilizados pelo professor, no intuito de promover a argumentação em sala de aula.


\subsection{Promovendo a argumentação}
\label{subsec: transcPromoArg}
Nesta seção utilizou-se duas ferramentas para classificar as perguntas feitas pelo professor dentro das categorias e em conjunto com os padrões de interações adotados pelos referenciais em seguida verificou-se o estabelecimento de uma cultura de sala de aula favorável à argumentação conforme sugere \cite{TELES:2021}.

Em \cite{SASSERON:2012} tem-se abordado um instrumento analítico para classificar as perguntas do professor dentro da perspectiva de um ensino promotor da Alfabetização Científica. Nele propõe-se categorias para as perguntas feitas pelo professor de Ciências em aulas investigativas.

No estudo dirigido por \cite{MORTIMER:2002}, padrões de interações surgem mediante a dinâmica que a sala de aula vai assumindo conforme professores e alunos alternam turnos de fala, o estudo cita o mais comum entre estes padrões as tríades I-R-A (iniciação do professor, resposta do aluno, avaliação do professor). A medida que esta dinâmica torna-se complexa, ocorre a formação de novos padrões como: I-R-P-R\ldots ou I-R-F-R-\ldots. Adotaremos a mesma notação da referência ao assinalar por P a ação discursiva que permite o \emph{prosseguimento} da fala do estudante, E para \emph{elicitação} e F um \emph{feedback} para que o estudante elabore mais a sua fala.

Retomando a aula anterior o professor inicia com uma pergunta exploratória sobre o processo, tem por objetivo estimular os alunos a relacionar ideias com dados e observações
\begin{itemize}
    \item 3. [Pr]: Você chegou a ler? A que conclusão vocês chegaram?
\end{itemize}
esta ação permite a criação e explanação de hipóteses como se observa nas respostas dos alunos [J] e [F]:
\begin{itemize}
    \item 4. [J]: Que as micro-ondas estão\ldots ou fazem uma grande agitação e\ldots elas passam essa\ldots agitação para o alimento nas várias formas e com essa agitação o alimento se aqueça\ldots fique com a
    temperatura maior\ldots
    \item 5. [F]: Aí vai aquecendo a superfície\ldots
    \item 6. [J]: Aquecendo\ldots agitando\ldots a superfície e passando através das partículas do alimento
    para todas as outras\ldots entrando para o centro do alimento\ldots
\end{itemize}
Com a próxima pergunta, o professor estabelece uma tríade do tipo I-R-F
\begin{itemize}
    \item 7. [Pr]: Esse processo é o cozimento?
\end{itemize}
Tem por intenção trabalhar os significados no desenvolvimento da estória científica. Faz uso de uma pergunta que exige raciocínio por parte dos alunos, buscando confrontar o que já conhecem sobre o fenômeno de aquecimento e o que coletaram de dados na atividade, com as hipóteses criadas.
\begin{itemize}
    \item 8. [J]: É\ldots constantemente\ldots as ondas estão dando agitação para as partículas da superfície dos alimentos\ldots estas vão dar para as mais de dentro e estas para as mais de dentro até ficar cozido.
    \item 9. [Pr]: Explica melhor\ldots o micro-ondas produz as ondas e quem irá sentir essas ondas? Quem irá interagir com as ondas do forno?
\end{itemize}
Até a resposta do aluno [J] pode-se dizer que a natureza do discurso estabelecido nesta etapa é interativo dialógico, mas há uma mudança perceptível para o discurso interativo de autoridade quando o professor fecha a pergunta em \emph{"...quem irá sentir essas ondas? Quem irá interagir com as ondas do forno?"}. O padrão de interação também evolui para um não triádico I-R-F-R-F.
\begin{itemize}
    \item 10. [J]: A água\ldots acho que a água\ldots
    \item 11. [Aluno 8]: Professor\ldots eu coloquei parecido com o [J]\ldots eu coloquei que as ondas interagem diretamente com o alimento\ldots não interagem com o recipiente ou com o ar\ldots que tá lá dentro\ldots então essa energia de agitação das moléculas do alimento vai ser maior\ldots que a energia\ldots das moléculas do ar que tá no forno normal [a gás]\ldots então tem mais diferença de temperatura\ldots vai ter e mais propagação de calor\ldots então vai evaporar mais água também e vai ficar mais seco.
    \item 12. [Pr]: Pera aí\ldots O [aluno 8] colocou uma nova situação: ele falou de moléculas de alimentos\ldots quem são as moléculas que basicamente constituem um alimento?
    \item 13. [Aluno 9]: Água\ldots
\end{itemize}
Neste ciclo o professor obtém um ponto chave na descrição do fenômeno pelas falas do aluno [J], [Aluno 8] e depois [aluno 9], o tipo do discurso é iterativo dialógico de autoridade, e tem por intenção dar foco ao que é necessário para que haja aquecimento no forno micro-ondas, destaca a exposição do [aluno 8] e da sequência a padrões de interação basicamente do tipo I-R-F e I-R-F-R-E, dessa forma vai auxiliando na construção do entendimento em conjunto com os estudante e sempre buscando tecer este entendimento a partir das ideias apresentadas pelos próprios estudantes.
Este procedimento se repete até que atinge uma confirmação por parte dos alunos, a partir dai o foco muda para:\emph{"\ldots a que temperatura chega o aquecimento dos alimento no micro-ondas"}
\begin{itemize}
    \item 19. [Pr]: Ótimo! Conta pra mim João, a que temperatura a água começa a evaporar? A que temperatura ela vai entrar em ebulição?
    \item 20. [J]: 100 graus\ldots
    \item 21. [Pr]: 100 graus Celsius\ldots agora eu pergunto o seguinte: será que essa micro-onda vai interagir com uma molécula de proteína?
    \item 22. [J]: As moléculas de água\ldots
    \item 23. [Pr]: O aluno 8 falou outra coisa importante: que a temperatura que o alimento foi submetido no micro-ondas é maior que a do que foi submetido no forno a gás.
\end{itemize}
Aqui o professor faz uso de perguntas que tem por aspectos discursivos a criação de um problema dentro da "estória científica", em geral são perguntas centrada na pessoa, buscando extrair o que o(s) aluno(s) sabe(m) sobre o processo além de sedimentar o aprendizado das questões anteriores.
\begin{itemize}
    \item 14. [Pr]: Água e que mais? Alimento é constituído de que? 
    \item 15. [Aluno 10]: Amido\ldots carboidratos\ldots e outras coisas.
    \item 16. [Pr]: Será que as micro-ondas interagem como um todo?\ldots em todas as moléculas? Será que elas interagem com todas?
    \item 17. [Aluno 5]: Acho que são com as da água, né?
    \item 18. [J]: É aí as moléculas de água passam para as outras moléculas do alimento.
    \item 24. [Aluno 11]: Eu discordo\ldots
    \item 25. [Pr]: Diga\ldots
    \item 26. [Aluno 11]: Se a molécula de água evapora a 100 graus, o máximo que ela vai ficar é até 100 graus Celsius\ldots no micro-ondas. Depois ela vai evaporar\ldots e no forno tem uma temperatura maior porque ele aquece todas as moléculas\ldots não só as de água\ldots A gente quando abre um micro-ondas vê um monte de vapor\ldots e no forno sente um bafo\ldots um ar quente\ldots
    \item 27. [Pr]: O forno que você falou chega a que temperatura\ldots que você vê escrito no botão?
    \item 28. [Aluno 12]: No meu forno tá escrito baixo, médio\ldots
    \item 29. [Pr]: Tá legal\ldots
    \item 30. [Aluno 11]: É 250, 300 graus Celsius\ldots    
\end{itemize}
Na sequência acima temos a formação de um padrão do tipo I-R-F-R-R-P-F\ldots o interessante nesta parte é que não há um consenso ainda sobre a que temperatura a água chega no micro-ondas, o [aluno 11] traz para a discussão o limite para o estabelecimento do ponto de ebulição ao nível do mar e o professor por sua vez utiliza desse argumento para finalizar as discussões
\begin{itemize}
    \item 31. [Pr]: É dá pra notar que no forno a gás a temperatura interna é muito maior, porque ele funciona, como vocês disseram, aquecendo tudo e no micro-ondas só a água\ldots para aquecer o resto. Agora só falta\ldots um instante pessoal\ldots quem ficou de ver como funciona os fornos de micro-ondas que também douram os alimentos?
    \item 32. [Aluno 13]: Eu\ldots tá\ldots li no catálogo que ele tem uma resistência dentro que após\ldots cozinhar é ligada para aquecer\ldots
    \item 33. [Pr]: Esse forno então é um tipo misto que funciona como micro-ondas e depois como forno elétrico\ldots tudo bem?
\end{itemize}
Por fim o professor fecha o ciclo utilizando a abordagem não interativa e de autoridade, não havendo mais intervenções.

Em resumo percebe-se que ocorreu um \emph{ciclo} na medida em que a atividade vai se desenvolvendo, este ciclo evolui de uma abordagem interativa dialógica, em que o professor promove discussões entre os grupos a fim de provocar nos alunos uma imersão na proposta didática e o desenvolver da estória científica, utiliza para isso padrões basicamente do tipo simples (I-R-F triádicos), considera o que os alunos tem a dizer e pede para destacar o que não compreenderam bem, em seguida o discurso do professor passa da abordagem interativa dialógica para a interativa de autoridade, os padrões de interação tornam-se mais complexos do tipo (I-R-F-R-F-P-E\ldots) e etc., esta é uma etapa longa e vai desde explorar as ideias dos alunos; fazê-los refletir sobre o que observaram, o que é previsto, o que teorizaram e tomaram por hipóteses, além de direcionar o discurso e os posicionamentos para a questão central da atividade, por último o ocorre o fechamento da atividade com uma abordagem não interativa de autoridade, o professor sintetiza o produto da atividade de forma expositiva, não havendo mais interações.

Baseado no estudo de \cite{TELES:2021} uma cultura de sala de aula favorável à argumentação, foi observada a partir da presença de alguns dos aspectos centrais da pesquisa mencionada, tendo como um dos fatores a \emph{dimensão temporal e processual}. Neste aspecto destaca-se que
\begin{citacao}
    ``\ldots a aprendizagem de argumentação científica ocorre de forma processual. o principal fator que contribuiu para aumentar a interação entre os(as) estudantes e a presença de contra-argumentos foi os(as) professores(as) colocarem questões mais abertas. \Ibidem[p. ~5]{TELES:2021}''
\end{citacao}
Dessa forma, as interações observadas nesta análise foram proporcionadas pelas duas questões iniciais propostas pelo professor, uma vez que são questões abertas (principalmente a questão que pede para os alunos explicarem como funcionam o aquecimento nos diversos tipos de eletrodomésticos).
Um outro fator relevante, associado dessa vez a \emph{dimensão social e coletiva} pode ser observado conforme o estudo
\begin{citacao}
    ``\ldots por meio das interações discursivas, estudantes influenciam-se mutuamente, contribuindo com a argumentação e a construção coletiva de conhecimentos. Do mesmo modo, os conhecimentos construídos pelo grupo influenciam a argumentação individual de cada estudante. \Ibidem[p. ~6]{TELES:2021}''
\end{citacao}
Isto é visto desde o início quando os alunos [J], [E] e [F] se questionam sobre \emph{o que é radiação}, buscando sempre obter as respostas de forma consensual e coletiva e a medida que validam uma etapa a próxima é construída com base nos progressos alcançados em etapas anteriores.

% \section{Qualidade da Argumentação}
% \label{sec: qualisArg}

% Nesta seção avaliaremos a qualidade da argumentação estabelecida nesta análise, utilizando como referenciais os trabalhos de \cite{SAMPSON:2006,SAMPSON:2012,SANDOVAL:2007}.

% \subsection{Formas}
% \cite{SAMPSON:2012}
% A conversa centrou-se na geração ou validação de alegações
% ou explicações

% os participantes usaram evidências para apoiar e desafiar as
% ideias ou dar sentido ao fenômeno sob investigação

% trechos:
% J: Não ... o micro-ondas a gente pode controlar a intensidade e o tempo ...

% \subsection{Conteúdos}
% \cite{SAMPSON:2012}
% Os participantes basearam as suas decisões ou ideias sobre
% estratégias de raciocínio inadequadas,

% trechos:
% J: A gente coloca ... da radiação ... como a gente faz? É irradiação ou radiação?
% E: Eu acho a luz ... a onda produz calor ... que produz a luz ...
% J: Sabe porque eu não acho que é filamento, senão seria resistível e não radiação ...

% Em uma pesquisa sobre o Desenvolvimento e Validação da Avaliação da Argumentação Científica em Sala de Aula - \emph{em inglês:} ASAC \cite{SAMPSON:2012}, traz uma ferramenta que pode ser utilizada pelo professor para estabelecer ASACs, nela a definição do constructo a ser mensurado durante a atividade, é tido como premissa para este desenvolvimento o que pode contribuir substancialmente para a qualidade da argumentação, neste sentido esta pesquisa adota uma visão de argumentação, similar a que foi estabelecida na atividade analisada nas  seções anteriores
% \begin{citacao}
%     ``adotamos uma visão de argumentação como um processo onde "diferentes perspectivas são consideradas e avaliadas a fim de obter-se um consenso que ditará o rumo das ações" \apud{DRIVER:2000}{SAMPSON:2012}. Esta visão de argumentação privilegia a colaboração em detrimento da competição e sugere que atividades promotoras de argumentação, podem fornecer um contexto onde indivíduos são capazes de usar as ideias uns dos outros para construir uma compreensão compartilhada de um fenômeno em particular, á luz das experiências vividas. \Ibidem[p. ~238]{SAMPSON:2012}'' -- tradução nossa\footnote{
%         we adopted a view of argumentation as a process where “different perspectives are being examined and the purpose is to reach agreement on acceptable claims or course of actions” (Driver et al., 2000, p. 291). This view of argumentation stresses collaboration over competition and suggests that activities that promote argumentation can provide a context where individuals are able to use each other’s ideas to construct and negotiate a shared understanding of a particular phenomenon in light of past experiences and new information
%         }  
% \end{citacao}