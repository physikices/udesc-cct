\chapter{Análise das Interações Discursivas}
\label{chap: iteracoes-habilidades}

% Nesta atividade, analisou-se a(s) interação(s) professor-aluno(s); aluno(s)-professor e aluno-aluno(s), bem como ocorreram estas interações do ponto de vista dos discursos entre os atores envolvidos, para tanto, utilizou-se como recurso a ferramenta dos autores \cite{MORTIMER:2002} em uma transcrição de uma aula sobre o funcionamento do forno micro-ondas \emph{(Tentando Explicar o Funcionamento do Forno Micro-Ondas)}. A transcrição é dividida em 02 duas cenas:
% \begin{itemize}
%     \item cena-1A: \emph{Tentando explicar o funcionamento do forno de micro-ondas}; e
%     \item cena-1B: \emph{Comparando o funcionamento dos fornos convencionais e de micro-ondas}.    
% \end{itemize}
% A metodologia utilizada para esta análise, consiste na caracterização, classificação e qualificação dos discursos apresentados na transcrição, segundo a abordagem sugerida pelos autores. Esta atividade foi desenvolvida ao longo de três aulas que ocorreram entre os dias 19/04 e 28/04, envolvendo:
% \begin{enumerate}[label=\alph *)]
%     \item Leitura completa do artigo e elaboração de questões para discussão em sala de aula - \emph{19/04};
% \end{enumerate}


\section{Proporcionando um ambiente de ensino construtivo}
\label{sec: analise-transcricao}
A transcrição inicia-se com os alunos retomando as questões apresentadas pelo professor numa aula anterior trazidas como ponto de partida para orientar as discussões que se seguem
\begin{enumerate}[label=\roman *)]
    \item Relacione abaixo os aparelhos e/ou dispositivos que podem ser usados em uma casa, destinados a provocar aquecimento;
    \item Agrupe-os e explique como eles funcionam.
\end{enumerate}
Identifica-se previamente uma certa aproximação da proposta do professor com a abordagem didática do \textbf{Ensino por Investigação}. Na perspectiva de \citeonline{SASSERON:2015} o Ensino por Investigação tem por objetivos, \emph{``levar os estudantes a realizarem investigação e de desenvolver entre os estudantes um entendimento sobre o que seja a investigação científica''}. Uma característica desta abordagem, reside em promover o papel ativo do aluno na construção do entendimento sobre os conhecimentos científicos, esta ação relaciona-se com a(s) intenção(ões) do professor de oportunizar situações que desenvolvam o protagonismo em sala de aula por parte dos estudantes, uma vez que é transferida a eles a responsabilidade de relacionar, classificar, agrupar e por fim explicar o funcionamento de cada aparelho encontrado em seu ambiente domiciliar. \cite{MORTIMER:2002} sintetiza as intenções do professor em seis categorias inter-relacionadas, tendo uma delas por foco \emph{engajar os estudantes, intelectualmente, no desenvolvimento inicial da "estória científica"}. Neste sentido, estas pesquisas concordam com o papel do professor como o
\begin{citacao}
    ``[..]de fazer com que a turma se engaje com as discussões e, ao mesmo tempo em que travam contato com fenômenos naturais, pela busca de resolução de um problema, exercitam práticas e raciocínios de comparação, análise e avaliação bastante utilizadas na prática científica.'' \cite[p. 58]{SASSERON:2015}
\end{citacao}

Partindo-se deste pressuposto e lendo na transcrição as falas dos alunos, notou-se que grande parte das investigações giraram em torno da natureza de funcionamento do forno de micro-ondas, onde os alunos são convidados indiretamente a  obter uma melhor compreensão dos conceitos físicos envolvidos na questão. Inicialmente o fazem por meio do diálogo entre os pares, o trecho a seguir ilustra um pouco desta dinâmica:
\begin{itemize}
    \item 2. [J]: A gente coloca ... da radiação ... como a gente faz? É irradiação ou radiação?
    \item 3. [E]: Eu acho que é i
    \item 4. [F]: Vou procurar no dicionário ... [pega na sua mala um dicionário em edição de bolso]    
\end{itemize}
Se tratando de conceitos científicos já consolidados e estruturados, é fundamental para o estudante estar diante do termo preciso atribuído ao conceito. Pela transcrição não é possível afirmar se o(a) aluno(a) [J] o faz despretensiosamente ao buscar num primeiro momento, delimitar o termo correto antes mesmo de apropriar-se do conceito, mas ao estabelecê-lo, consolida um dos três eixos estruturantes da \textbf{Alfabetização Científica} apresentados na pesquisa e que tem relação direta com o Ensino por Investigação
\begin{citacao}
    ``(a) a compreensão básica de termos e conceitos científicos, retratando a importância de que os conteúdos curriculares próprios das ciências sejam debatidos na perspectiva de possibilitar o entendimento conceitual;'' \cite[p. 57]{SASSERON:2015}
\end{citacao}
Após lerem a definição da palavra \emph{irradiação} encontrada no dicionário, os alunos começam a se questionar,
\begin{itemize}
    \item 6. [F]: “ato ou efeito de irradiar, bombardeio de uma substância por um feixe de partículas”
    \item 7. [J]: Mas ... até aí ... 
    \item 8. [E]: Então vê radiação ...
\end{itemize}
Porém não ficam satisfeitos com a definição do dicionário, precisam que apareça na definição elementos como \emph{filamento} ou algo de que já conhecem sobre aquecimento, sendo assim buscam pela definição da outra palavra, \emph{radiação}
\begin{itemize}
    \item 11. [J]: Como assim, ele provoca luz? ... mas o aquecimento, o calor ... de onde vem o calor? Da luz?
    \item 12. [E]: Do feixe de luz
    \item 13. [J]: Será ... assim? Eu não sei ... 
    \item 14. [E]: Eu acho que também deve ser um tipo de filamento
    \item 15. [F]: Achei ... “radiar: emitir ondas e energia calorífica, luminosa, etc. Cintilar, resplandecer”.
\end{itemize}
É claro que os significados das palavras \emph{radiação} e \emph{irradiação} assim encontradas no dicionário, contribuem muito pouco para elucidar os conceitos físicos por trás da tecnologia de aquecimento utilizada em micro-ondas, pelo contrário, arrisco a dizer que podem até confundir os alunos\footnote{Radiação e Irradiação em Física, diz respeito ao transporte de energia térmica na forma de calor, enquanto um está relacionado ao transporte em si, o outro está relacionado às circunstâncias em que ocorre este transporte. O aquecimento no forno de micro-ondas se dá em termos do trabalho termodinâmico e não do calor.}, mas se essa conduta for bem explorada sob a supervisão do professor, pode tornar os resultados didaticamente mais interessantes, como prevê a pesquisa
\begin{citacao}
    ``[...]o ensino por investigação exige que o professor valorize pequenas ações
    do trabalho e compreenda a importância de colocá-las em destaque como, por
    exemplo, os pequenos erros e/ou imprecisões manifestados pelos estudantes, as
    hipóteses originadas em conhecimentos anteriores e na experiência de sua turma,
    as relações em desenvolvimento.'' \cite[p. 58]{SASSERON:2015}
\end{citacao}
Este processo é viabilizado e sem demora os estudantes consultam a orientação do professor como segue
\begin{itemize}
    \item 27. [J]: Eu vou ler de novo ... o que diz sobre radiação pra gente pensar o que é .... é o aquecimento através de ondas, não é? Por isso é que chama micro-ondas ... a luz eu sei que tema luz ... a onda é aquela tal ... mas ... quando a gente colocar um prato não aquece por igual ... as vezes uma parte fica fria e a outra ...
    \item 29. [J]: É? ... As ondas têm irregularidades?
    \item 30. [E]: Chama o professor ... mostra pra ele ...
    \item 31. [J]: Professor tá difícil ... essa coisa de micro-ondas ...
\end{itemize}
A resposta do professor revela uma segunda intenção
\begin{itemize}
    \item 32. [Pr]: Deixa eu ver ... ajudar na discussão ... Eu quero cozinhar uma carne, por exemplo, pra isso eu posso dispor do fogão ... a combustível ... a gás e de um forno de micro-ondas ... a primeira coisa ... é o tempo de cozimento eles são iguais?
\end{itemize}
De acordo com \cite{MORTIMER:2002}, as questões levantadas pelo professor tem por objetivo explorar melhor a visão e entendimento dos estudantes sobre as diferentes formas de aquecimento devido aos diferentes tipos de eletrodoméstico, usa portanto o confronto do que já é conhecido pelo estudantes, com o tema da investigação.

Nesta intervenção, já começam a aparecer algumas classes de \textbf{Abordagem Comunicativa} caracterizadas pelos diferentes padrões de iterações entre professor-aluno, aluno-aluno e vice-versa. \cite{MORTIMER:2002} caracteriza estas interações em duas dimensões extremas: o \emph{discurso dialógico ou de autoridade} e o \emph{discurso interativo ou não-interativo.}
Em resposta as questões do professor o aluno(a) [J] traz o que lhe é natural, experimentado no cotidiano
\begin{itemize}
    \item 33. [J]: Não ... o micro-ondas a gente pode controlar a intensidade e o tempo ...
\end{itemize}
A abordagem comunicativa dialógica se manifesta na resposta do professor ao considerar ao ponto de vista do estudante
\begin{itemize}
    \item 34. [Pr]: Ótimo ... o micro-ondas doura as coisas?
\end{itemize}
A presença da palavra \emph{"Ótimo"} na fala do professor, sugere que os alunos estão no caminho certo, ainda que os argumentos como intensidade e controle do tempo de cozimento, não são características inerentes apenas ao forno de micro-ondas. Dessa forma os estudantes sentem-se motivados a expor suas ideias e são até advertidos a organizarem suas falas um por vez.

Ainda sobre esta fala, ao perguntar se o micro-ondas doura as coisas, o professor direciona a atenção do grupo para os objetivos da investigação. Segundo \cite{SASSERON:2012} este tipo de pergunta é bem esperada dentro da abordagem do Ensino Investigativo e a categoriza como uma pergunta de foco e atenção, dessa forma o professor ajuda os estudantes a manterem-se no foco e prestar atenção aos detalhes da problemática.   

Com isso o discurso evolui para uma nova fase, a fase de formação de hipóteses.  


\section{Promovendo a argumentação}
\label{sec: analise-habilidades}