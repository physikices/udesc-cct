\chapter{Apresentação da Concedente}
A autora \citeonline[p. 5]{CARVALHOAMP:2012a} orienta os estagiários a conhecerem e descreverem a escola da seguinte forma:
\begin{citacao}
``O conhecimento de uma escola passa pelo conhecimento de sua parte física e humana.(...)mas principalmente com a sua sensibilidade (do estagiário). Ela é limpa? É um ambiente agradável fisicamente? Um membro da equipe diretora está sempre presente na escola?''\cite[p. 5]{CARVALHOAMP:2012a}
\end{citacao}
Neste sentido a tomada de conhecimento, por parte do estagiário, das dimensões: humana; organizacional, e contextual a que está inserida a unidade de ensino torna-se essencial para orientar a prática pedagógica do futuro professor e dar-lhes subsídios concretos para a tomada de ação, isto for feito através da análise do Plano de Desenvolvimento Institucional (PDI) do IFSC como segue.

\begin{verbatim}
[********************OBSERVAÇÕES EXTERNAS AO TEXTO***********************
"EM CONSTRUÇÃO A PARTIR DAQUI" - DESENVOLVER AS SEÇÕES DE MODO A DIALOGAR
COM O REFERENCIAL TEÓRICO
*************************************************************************]
\end{verbatim}

\section{Caracterização da Escola}
\begin{verbatim}
[********************OBSERVAÇÕES EXTERNAS AO TEXTO***********************
"Esta seção ainda não está bem desenvolvida e estruturada, carece de mais
informações como por exemplo dimensionamento da infraestrutura e etc, além
de confrontar o que está nos documentos com o referencial teórico.
*************************************************************************]
\end{verbatim}

O Instituto Federal de Santa Catarina (IFSC), Câmpus Joinville, propõe-se a desenvolver um trabalho contínuo na busca de parcerias com a comunidade para divulgação de novas formas de se fazer educação profissionalizante, baseadas no tripé ensino, pesquisa e extensão.

Atualmente, o câmpus atende aproximadamente 1,8 mil alunos em cursos presenciais e funciona nos três turnos. A infraestrutura é composta por salas de aula, laboratórios, laboratórios de informática, biblioteca informatizada, auditório, cantina e ginásio esportivo.



\section{Histórico}
Criada em Florianópolis em 23 de setembro de 1909, por meio do Decreto nº 7.566 as Escolas de Aprendizes Artifices, para o ensino profissional primario e gratuito, ainda no mandato do então presidente Nilo Peçanha. O IFSC é uma instituição de ensino pública federal e atua na oferta de educação profissional, científica e tecnológica, oferecendo cursos nos níveis de qualificação profissional, educação de jovens e adultos, cursos técnicos, superiores e de pós-graduação.

Passa a atuar na cidade de Joinville após um convênio com o Hospital Dona Helena, em 1994, dando início ao funcionamento do curso técnico em Enfermagem. Nesta parceria, o Hospital cedeu as instalações e equipamentos e o IFSC disponibilizou o quadro de docentes, a concepção, o desenvolvimento e a implementação da estrutura curricular do curso.

O Ministério da Educação (MEC), por meio de sua Secretaria de Educação Profissional e Tecnológica (Setec/MEC), criou no final de 2005, o Plano de Expansão da Rede Federal de Educação Profissional, com objetivo de ampliar a presença destas instituições em todo o território nacional, isso possibilitou a transformação da então Gerência Educacional de Saúde de Joinville em Unidade de Ensino, em agosto de 2006. A construção das instalações próprias do Câmpus Joinville do IFSC permitiu a ampliação da oferta de cursos na área industrial, com os cursos técnicos em Eletroeletrônica e Mecânica Industrial (atualmente Mecânica).

A oferta dos novos eixos veio ao encontro do perfil industrial da cidade, formado por grandes conglomerados do setor metal-mecânico, químico, plásticos, têxtil e de desenvolvimento de software.

Desde sua inauguração, o Câmpus Joinville busca a ampliação de sua área física e aumento da oferta de cursos. No segundo semestre de 2009, ocorreu a implantação dos cursos superiores de tecnologia em Gestão Hospitalar e Mecatrônica Industrial (que deu lugar aos cursos de Engenharia Elétrica e Engenharia Mecânica).

Em 2011, iniciou as atividades dos cursos técnicos integrados ao Ensino Médio em Eletroeletrônica e Mecânica.

A fim de fortalecer a área de Saúde e Serviços, em 2016, teve início o curso de especialização técnica em Saúde do Idoso, para quem já concluiu o curso técnico em Enfermagem. No segundo semestre de 2019, começou o curso superior de bacharelado em Enfermagem.

Em outra frente, o IFSC também trabalha pela ampliação dos eixos tecnológicos e oferta de cursos em outras áreas. Assim, no primeiro semestre de 2020, teve início o curso técnico concomitante em Teatro.

\section{Projeto Político Pedagógico}
\begin{verbatim}
[********************OBSERVAÇÕES EXTERNAS AO TEXTO***********************
"Esta seção e subseções ainda não estão bem desenvolvidas e estruturadas,
ainda preciso embasá-las melhor para poder confrontar com o referencial
teórico.
*************************************************************************]
\end{verbatim}

\subsection{Concepções Norteadoras}
O Projeto Pedagógico Institucional do IFSC toma como ponto de partida o marco referencial teórico-metodológico elaborado e construído de forma coletiva pelos integrantes da comunidade escolar. As concepções norteadoras explicitadas neste documento os fundamentos básicos que orientarão a formulação de diretrizes, políticas e projetos da instituição, e atuarão como bases da unidade do IFSC em seu processo de planejamento, execução e avaliação dos planos de ensino, pesquisa e extensão.
\subsection{Apresentação do PPI}
O Projeto Pedagógico Institucional do IFSC toma como ponto de partida o marco referencial teórico-metodológico elaborado e construído de forma coletiva pelos integrantes da comunidade escolar. As concepções norteadoras explicitadas neste documento constituirão os fundamentos básicos que orientarão a formulação de diretrizes, políticas e projetos da instituição, e atuarão como bases da unidade do IFSC em seu processo de planejamento, execução e avaliação dos planos de ensino, pesquisa e extensão. \cite{IFSC:2020}
\subsection{Sistema de Avaliação}