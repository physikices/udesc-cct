%=========|CABECALHO|=========>
\thispagestyle{empty}
\center
\begin{minipage}[!]{\linewidth}
	\begin{minipage}[!]{.19\linewidth}
		\includegraphics[width=\textwidth]{img/Marca Joinville Vertical RGB-01.jpg}				
	\end{minipage}
	\begin{minipage}[!]{.8\linewidth}
		\center
		\textsf{
			\large{
				\instituicao \\ \vspace{0.1cm}
				\centro \\ \vspace{0.17cm}
				\departamento
				%\disciplina
			}	
		}		
	\end{minipage}
	\center
	\rule{\textwidth}{1pt}	
	\textsc{\autor} \\
	\today \\ 
	\rule{\textwidth}{1pt}	
\end{minipage}
%\vspace{1cm}
\begin{center}
   \textbf{\titulo}\\  
   \rule{.3\textwidth}{0.1pt}
\end{center}
\renewcommand{\thesection}{\Roman{section}}
%=========|END-CABECALHO|=====>


%============| INÍCIO |=======================================>
%--------------| Q01 |---------------------------------------->
\begin{prob}
	Mostre que:
	\begin{enumerate}[label=\alph *)]
		\item $1\textnormal{ ano-luz}=9,46\times 10^{12}km$
		\item $1\textnormal{ parsec}=3,26\textnormal{ anos-luz, i.e. }3,08\times10^{13}km$
	\end{enumerate}	
\end{prob}

\textcolor{red}{
	\begin{sol}
		Considerando $v_l=299.792.458m/s$ para a velocidade da luz e $t=31.557.600s$ para o período de um ano
		\begin{enumerate}[label=\alph *)]
			\item $1\textnormal{ ano-luz}=(v_l)t=(3,0\times 10^{8}m/s)(3,1\times 10^{7}s)=9,46\times 10^{15}m=9,46\times 10^{12}km$
		\end{enumerate}
	\end{sol}
}







