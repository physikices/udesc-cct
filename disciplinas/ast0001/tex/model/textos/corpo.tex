%=========|CABECALHO|=========>
\thispagestyle{empty}
\center
\begin{minipage}[!]{\linewidth}
	\begin{minipage}[!]{.19\linewidth}
		\includegraphics[width=\textwidth]{img/Marca Joinville Vertical RGB-01.jpg}				
	\end{minipage}
	\begin{minipage}[!]{.8\linewidth}
		\center
		\textsf{
			\large{
				\instituicao \\ \vspace{0.1cm}
				\centro \\ \vspace{0.17cm}
				\departamento
				%\disciplina
			}	
		}		
	\end{minipage}
	\center
	\rule{\textwidth}{1pt}	
	\textsc{\autor} \\
	\today \\ 
	\rule{\textwidth}{1pt}	
\end{minipage}
%\vspace{1cm}
\begin{center}
   \textbf{\titulo}\\  
   \rule{.3\textwidth}{0.1pt}
\end{center}
\renewcommand{\thesection}{\Roman{section}}
%=========|END-CABECALHO|=====>


%============| INÍCIO |=======================================>
%--------------| Q01 |---------------------------------------->
\begin{prob}
	Seja
	\begin{align}
		\vec{r}:\left\{\begin{matrix}
		x=r\cos\varphi \\
		y=r\sen\varphi
	\end{matrix}\right.\qquad \textsl{e}\quad\vec{r}=x \hat{i}+y \hat{j}
	\end{align}
encontrar $\hat{r}$ e $\hat{\varphi}$ em termos de $\hat{i}$ e $\hat{j}$,

\begin{enumerate}[label=\alph *)]
	\item Utilizando
		\begin{align}
			\hat{r}&=\frac{\frac{d\vec{r}}{dr}}{|\frac{d\vec{r}}{dr}|}				\qquad \textsl{e}\quad\hat{\varphi}=\frac{\frac{d\vec{r}}					{d\varphi}}{|\frac{d\vec{r}}{d\varphi}|}
		\end{align}
	\item Demonstre que $\dot{\vec{r}}=\dot{r}\hat{r}+r\dot{\varphi}\hat{\varphi}$.
	\end{enumerate}
\end{prob}

\textcolor{red}{
	\begin{sol}
		\begin{enumerate}[label=\alph *)]
			\item Visto que o vetor $\vec{r}$, pode ser escrito em coordenadas polares como $\vec{r}=r\cos\varphi\hat{i}+r\sen\varphi \hat{j}$ tem-se:
			\begin{align}
				\frac{d\vec{r}}{dr}&=\frac{d}{dr}\left(r\cos\varphi\hat{i}+r\sen\varphi\hat{j}\right)\nonumber\\ 
				\frac{d\vec{r}}{dr}&=\cos\varphi\hat{i}+\sen\varphi\hat{j}
			\end{align}
	cujo o módulo é \cite{2004}
			\begin{align}
				\left\vert\frac{d\vec{r}}{dr}\right\vert&=\sqrt{\cos^2\varphi+\sen^2\varphi}=1
			\end{align}
	e
			\begin{align}
				\frac{d\vec{r}}{d\varphi}&=\frac{d}{d\varphi}\left(r\cos\varphi\hat{i}+r\sen\varphi\hat{j}\right)\nonumber\\
				\frac{d\vec{r}}{d\varphi}&=-r\sen\varphi\hat{i}+r\cos\varphi\hat{j}
			\end{align}
	de módulo
			\begin{align}
				\left\vert\frac{d\vec{r}}{d\varphi}\right\vert&=\sqrt{r^{2}\sen^2\varphi+r^{2}\cos\varphi}=r
			\end{align}
	e por fim obtemos
			\begin{align}
				\hat{r}&=\cos\varphi\hat{i}+\sen\varphi\hat{j}\\
				\hat{\varphi}&=-\sen\varphi\hat{i}+\cos\varphi\hat{j}.
			\end{align}
			\item Sendo $\vec{r}$ definido por $\vec{r}\rightarrow\vec{r}(r,\varphi)$ então $\dot{\vec{r}}$ é dado, por Leibniz
			\begin{align}
				\parder{}{t}\vec{r}(r,\varphi)&=\parder{\vec{r}}{r}\frac{dr}{dt}+\parder{\vec{r}}{\varphi}\frac{d\varphi}{dt}
			\end{align}
	é claro que $\partial_{r}$ e $\partial_\varphi$ de $\vec{r}(r,\varphi)$ é simplesmente $d\vec{r}/dr$ e $d\vec{r}/d\varphi$ respectivamente, além do mais $\dot{r}=dr/dt$ e $\dot{\varphi}=d\varphi/dt$, então substituindo os resultados do item a) temos
			\begin{align}
				\dot{\vec{r}}=\parder{}{t}\vec{r}(r,\varphi)&=\left(\cos\varphi\hat{i}+\sen\varphi\hat{j}\right)\dot{r}+\left(-r\sen\varphi\hat{i}+r\cos\varphi\hat{j}\right)\dot{\varphi}\nonumber\\
				&=\hat{r}\dot{r}+r\left(-\sen\varphi\hat{i}+\cos\varphi\hat{j}\right)\dot{\varphi}\therefore\nonumber\\
				\dot{\vec{r}}&=\dot{r}\hat{r}+r\dot{\varphi}\hat{\varphi}.
			\end{align}
		\end{enumerate}
	\end{sol}
}







