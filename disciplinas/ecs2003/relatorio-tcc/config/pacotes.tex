\usepackage[
	alf, 
	versalete, 
	abnt-emphasize = bf, 
	abnt-etal-list = 3, 
	abnt-etal-text = it, 
	abnt-and-type = &, 
	abnt-last-names = abnt,
	abnt-repeated-author-omit = yes
	]{abntex2cite}

%==========|CODIFICACAO|=================================================================>
\usepackage{lmodern}	       			% usa a fonte Latin Modern
\usepackage[T1]{fontenc}				% selecao de codigos de fonte.
\usepackage[utf8]{inputenc}		    % codificacao do documento (conversão automática dos acentos)
\hyphenation{bi-blio-gra-fia}		% controla hifenação
\usepackage{soul}					% for underlines
\usepackage{epigraph}
%==========|FIM-CODIFICACAO|=============================================================>

%==========|MATEMATICO|==================================================================>
\usepackage{amsmath}
\usepackage{amssymb, amsfonts, amsthm}
\usepackage{changepage,	threeparttable} 			
%==========|FIM-MATEMATICO|==============================================================>

%==========|GRAFICOS|====================================================================>
\usepackage{graphicx}			    % inclusão de gráficos

%----------|Cores|=======================================================================>
\usepackage{color}				    % controle de cores
%\usepackage[table]{xcolor}			% tabelas coloridas

\definecolor{color_200_200_200}{RGB}{200,200,200}
\definecolor{blue}{RGB}{41,5,195}
%==========|FIM-GRAFICOS|================================================================>

%==========|LETTRINE|====================================================================>
\usepackage{lettrine}
\usepackage{GoudyIn}
%\usepackage{erewhon}				% ornamento da primeira letra de um parágrafo
\usepackage[x11names]{xcolor} 
\renewcommand{\LettrineFontHook}{\color{VioletRed4}\GoudyInfamily{}}
%\LettrineTextFont{\itshape}
\setcounter{DefaultLines}{3}
%==========|END-LETTRINE|================================================================>

%==========|PDF|=========================================================================>
\usepackage{lastpage}			    % usado pela Ficha catalográfica
%\usepackage{indentfirst}			% indenta o primeiro parágrafo de cada seção.
\usepackage{microtype} 			    % para melhorias de justificação
\usepackage{fancyhdr}				% cabeçalhos
\setlength{\headheight}{15.35403pt}
\usepackage{lscape}					% altera a orientação da saida PDF
%\usepackage{pdflscape}				% altera a orientação da saida PDF
\usepackage{setspace}
\usepackage{pdfpages}
\usepackage{adjustbox}
\usepackage{booktabs, multirow} 		% for borders and merged ranges|e estrutura multicoluna
\usepackage{lipsum}					% gerador de textos (dummy)
\usepackage{ragged2e}

%==========|FIM-PDF|=====================================================================>

%==========|BIBLIOGRAFIA|================================================================>
\usepackage{hyperref}
\hypersetup{
     	%pagebackref=true,
		colorlinks=true,       		% false: boxed links; true: colored links
    	linkcolor=blue,          	% color of internal links
    	citecolor=red,        		% color of links to bibliography
    	filecolor=magenta,      		% color of file links
		urlcolor=blue,
		%==========|meta dados-pdf|===============================================>
		pdfauthor={Rodrigo Nascimento},
		pdftitle={Relatório de Estágio Curricular II},
		pdfsubject={Ensino de Física},
		pdfkeywords={Estágio Supervisionado; Ensino Médio; Física.},
		bookmarksdepth=4
}
% \usepackage{todonotes}
%==========|FIM-BIBLIOGRAFIA|============================================================>

%==========|AMBIENTES|===================================================================>
%----------|Quadros|--------------------------------------------------------------------->
\newcommand{\quadroname}{Quadro}
\newcommand{\listofquadrosname}{Lista de quadros}
\newfloat[chapter]{quadro}{loq}{\quadroname}
\newlistof{listofquadros}{loq}{\listofquadrosname}
\newlistentry{quadro}{loq}{0}
\setfloatadjustment{quadro}{\centering}
\counterwithout{quadro}{chapter}
\renewcommand{\cftquadroname}{\quadroname\space} 
\renewcommand*{\cftquadroaftersnum}{\hfill--\hfill}
\setfloatlocations{quadro}{hbtp}

%----------|Exercicios|------------------------------------------------------------------>
\theoremstyle{definition}
\newtheorem{xca}{Problema}
\newenvironment{prob}{%
\par\noindent \begin{xca}}%
{\end{xca}\noindent\rule{\textwidth}{1pt}}
\newtheorem{sol}{Solução}
\newtheorem{theorem}{Teorema}[section]
\newtheorem{corollary}{Corolário}[theorem]
\newtheorem{lemma}[theorem]{Lema}

%----------|Comandos|-------------------------------------------------------------------->
\DeclareMathOperator{\sen}{sen}
\DeclareMathOperator{\tg}{tg}
\newcommand{\parder}[2]{\frac{\partial {#1}}{\partial {#2}}}

%----------|Ajustes|--------------------------------------------------------------------->
\setlength{\parindent}{1.3cm}		% tamanho do parágrafo
\setlength{\parskip}{0.2cm} 			% espaçamento entre paragrafos (\onelineskip)
%==========|FIM-AMBIENTES|===============================================================>