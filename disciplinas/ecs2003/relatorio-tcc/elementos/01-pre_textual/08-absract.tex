\begin{resumo}[Abstract]
    \begin{otherlanguage*}{english}
        The eclipse of the Sun of 1919 was fundamental in the development of physics and earns a high place in the history of science. Several players took part in this adventure. The most important are Einstein, Dyson, Eddington, the Sun, the Moon, Sobral, and Principe. Einstein's theory of gravitation, general relativity, had the prediction that the gravitational field of the Sun deflects an incoming light ray from a background star on its way to Earth. The calculation gave that the shift in the star's position was 1.75 arcseconds for light rays passing at the Sun's rim. So to test it definitely it was necessary to be in the right places on May 29, 1919, the day of the eclipse. That indeed happened, with a Royal Greenwich Observatory team composed of Crommelin and Davidson that went to Sobral, and that was led at a distance by the Astronomer Royal Frank Dyson, and with Eddington of Cambridge University that went to Principe with his assistant Cottingham. The adventure is fascinating, from the preparations, to the day of the eclipse, the data analysis, the results, and the history that has been made. It confirmed general relativity, and marked an epoch that helped in delineating science in the post eclipse era up to now and into the future. This year of 2019 we are celebrating this enormous breakthrough.
   
   
        \vspace{\onelineskip}
        \noindent 
       
        \textbf{Keywords}: Light deflection; General relativity; Eclipse 1919
    \end{otherlanguage*}
\end{resumo}