\chapter{Acompanhamento de Reuniões}
\label{cap: reunPedagogica}
\section{Conselho de Classe}

O Conselho de Classe do IFSC, é feito em duas etapas, a primeira tem a participação do aluno representante de classe, à este, é dada a oportunidade de expor de maneira geral as dificuldades que a classe possui com determinada disciplina. A segunda etapa, é restrita somente aos professores e cabem a eles, em conjunto com a equipe pedagógica, avaliar o desempenho geral de cada estudante e propor soluções. Segue a transcrição de alguns trechos da reunião observada.

\begin{enumerate}
	\item	Etapa com o aluno representante:
	\begin{enumerate}
		\item O coordenador do conselho iniciou a reunião pedindo a exposição da aluna [A] em termos das dificuldades enfrentada pelos alunos da turma, no quesito acompanhamento e participação das aulas.
		
		\item A aluna responde positivamente indicando que apesar do cenário (pandemia) a maioria dos alunos está conseguindo acompanhar as aulas e aponta que há dificuldade ainda na turma em participar mais ativamente das aulas.
		
		\item Em seguida a professora [B] interrompe comentando que este é um comportamento observado em outras turmas, pontua que as aulas virtuais tem trazido esta problemática, compreende que a dinâmica é completamente diferente e difícil ainda de ser superada.
		
		\item O Professor [C-Física] diz que, na turma dele, os alunos desta turma tem boa participação, boas notas e que no geral está tudo bem.
		
		\item A professora [D] diz que sente uma certa dificuldade mesmo na questão das interações (aluno-professor-turma), fala sobre a falta de troca (de interações) nas atividades síncronas, diz que a turma tem maturidade e é necessário fazer atividades em que os façam se expor mais durante as aulas.
		
		\item A Professora [E-Matemática]: Fala sobre a volta às aulas presenciais e alerta sobre a dificuldade que os alunos podem sentir no retorno as atividades sem consulta, fala que tem havido desistências e que a aluna tem razão na baixa participação das turmas e no protagonismo em sala de aula. Ressalta ainda que trata-se de uma turma que não teve aulas presenciais em todo o decorrer do curso ainda.
		
		\item O professor [F] avisa a aluna para alertar a turma sobre as atividades, que devem ser entregues.
		
		\item O professor [G-Ciências dos Materiais] comenta que as atividades de laboratório voltaram presencialmente a pouco tempo e os alunos se sentem animados pela visão dele.
		
Em geral a turma é bem avaliada mas são feitas ressalvas sobre a participação efetiva dos alunos nas aulas.
	\end{enumerate}
	\item Etapa restrita aos professores:
	\begin{enumerate}
		\item Cada professor inicia a sua análise com base em seus diários. Esta análise leva em consideração os resultados das avaliações feita por cada aluno, onde é verificado o desempenho deles refletido sobre as notas das avaliações, também é observado a presença e a participação de cada aluno nas aulas. Alunos que possuem desempenho insatisfatório em uma determinada disciplina, mas não nas outras, é ainda analisado, para efeito de comparação, outros critérios como faltas; participação e pontualidade nas entregas das atividades. Em todo caso, atividades de recuperação são discutidas e oportunizadas com cada aluno e turma em específico. Somente quando o aluno deixa de fazer também as atividades de recuperação, possuem faltas demais e/ou não conseguem desempenho satisfatório após a recuperação, é sugerido o acompanhamento e monitoração junto à equipe pedagógica da instituição.
	\end{enumerate}
\end{enumerate}
  


\section{Reunião Pedagógica}
