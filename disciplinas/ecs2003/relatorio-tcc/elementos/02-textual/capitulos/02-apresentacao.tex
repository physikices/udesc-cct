\chapter{Apresentação da Concedente}
\label{cap: aprConcedente}

Um estudo elaborado por \cite{CARVALHOAMP:2012a} recomenda aos estagiários a tomarem conhecimento do ambiente escolar primeiramente por intermédio da observação. Buscando entre uma observação e outra compreender a realidade a que estará inserido no decorrer de sua atuação profissional, em tese isso deve contribuir para orientar a prática pedagógica do futuro professor e fornecer-lhes dados subsidiários, para uma formação sólida e humanizada.
\begin{citacao}
``O conhecimento de uma escola passa pelo conhecimento de sua parte física e humana.(...)mas principalmente com a sua sensibilidade (do estagiário). Ela é limpa? É um ambiente agradável fisicamente? Um membro da equipe diretora está sempre presente na escola?'' \Ibidem[p. ~5]{CARVALHOAMP:2012a}
\end{citacao}
Adotando esta recomendação, dar-se-á por iniciada nossas atividades.

\section{Caracterização da Unidade Escolar}
De muros altos e pintados de verde a Giovani Pasqualini Faraco se mistura à paisagem urbana da região norte de Joinville, não passa por despercebida pois tudo em sua estrutura, ainda que de certa forma, escondida pelos muros; revela o padrão estrutural das unidades escolares públicas da região. Em tinta branca e em caixa alta é lido o nome da escola numa parte do muro, dessa forma os muros não apenas lhe conferem \emph{proteção}, mas também referência.

O acompanhamento dos muros nos leva a entrada principal, consiste num portão largo, aberto ao início e final de cada turno, é por onde passam a maioria dos alunos, um segundo portão feito de grade e estreito é visto mais adiante, este da acesso à secretaria. De clima agradável, muito bem equipada e limpa a secretaria é a primeira instância administrativa de que a comunidade tem acesso direto à escola, há sempre alguém de prontidão e não demora muito a atendê-lo. Folhetos no balcão e cartazes colados nos murais, informam as datas e atividades que a escola desenvolve em conjunto com a comunidade. Um caminho estreito e florido, liga a secretaria ao rol onde ficam; as salas dos(as) Assistentes Técnicos(os) Pedagógicas(os), da Direção e a dos Professores.

A unidade funciona nos três turnos disponíveis, atende desde o 1º ao 9º ano do Ensino Fundamental (EF) e do 1º ao 3º ano do Ensino Médio (EM).

\subsection{Histórico}
Fundada em 15 de fevereiro de 1938, a Escola Desdobrada Dona Francisca Quilômetro Cinco, fica situada junto à Rua Dona Francisca, número 4957, Bairro Santo Antônio -  Joinville/SC. Têm em seu nome o Patrono da instituição, Giovani Faraco.

Nascido em 05 de abril de 1915, filho de Biase Faraco e Maria Limongi Faraco, conclui o Ensino Primário na Escola São José, ingressando em seguida no Ginásio onde permanece como aluno interno até o ano de 1929. Em 1930 entra pro Seminário Central de São Leopoldo, onde permanece até o ano de 1935. Ao passo de sua jornada, lecionou latim e português em diversos estabelecimentos particulares. Possui boa desenvoltura musical e compôs várias partituras para o instrumento órgão. Falece em 10 de novembro de 1960, vítima de hipertensão arterial, deixando sua esposa Tusnelda Gomes Faraco e cinco filhos.

O decreto de n$^\circ$ 10138/70 renomeia a unidade para Grupo Escolar Giovani Pasqualini Faraco e em 08 de junho de 1993, recebe aprovação do Conselho Estadual de Educação para implantar o Ensino Médio através do parecer de n$^\circ$ 117/93. Atualmente integra a rede Estadual de Ensino sob o nome Escola de Educação Básica Giovani Pasqualini Faraco.

\subsection{Infraestrutura}
Tem por área construída 3114,50 $m^2$ e outros 13929,25 $m^2$ de área disponível. Conta com uma infraestrutura de; 17 salas de aulas  de 42 $m^2$ cada, cozinha, depósito para merenda escolar, depósito para produtos de limpeza, cantina, banheiros feminino e masculino, banheiros para professores, biblioteca, laboratório para as disciplinas de Física/Química e Biologia, laboratório de informática, sala ambiente para a disciplina de língua Portuguesa, arquivo inativo, sala para materiais de Educação Física, sala para materiais e equipamentos de: orientação escolar, direção, vídeo, artes e materiais para docentes.

A área descoberta é bem agradável e limpa, possui árvores ao seu redor criando um ambiente convidativo para o exercício da leitura, recreação ou até mesmo aulas diversificadas. Possui também uma pérgola onde realizam aulas de leitura. Para um contato maior com a terra, dispõe de uma horta escolar, onde os alunos podem fazer o reconhecimento de hortaliças e vegetais.

Possui pátio e quadras cobertas, onde são feitas as atividades culturais e esportivas, além das refeições no horário do recreio.

\subsection{Recursos humanos}
A equipe gestora da unidade é caracterizada pela Diretora e duas Assessoras que estão sempre disponíveis nas dependências do estabelecimento. Para a funcionalidade pedagógica e administrativa, conta com uma equipe técnica de Assistentes Técnicos(as) Pedagógicos(as) e Assistentes Educacionais, todos bem solícitos e acessíveis aos alunos, pais e estagiários.

\subsection{Perfil socioeconômico}
É situada perto de empresas locais e centros educacionais como: UNIVILLE, SENAI, IFSC e UDESC. Os estudantes que compõem o corpo discente, em geral, são de famílias pertencente à classe média, 80\% de etnia branca, 10\% negras, 5\% pardos e indígenas e outros 5\% não declarados. As famílias são de religião predominantemente cristã em que 60\% são evangélicos, 25\% católicos, 5\% luteranos, 5\% de raiz africana e 5\% não declarados. Ocupam-se das mais variadas funções, distribuídas em 30\% autônomos, 40\% funcionários das indústrias da região, 15\% exercem atividades comerciais, 5\% prestam serviços e 10\% não declararam.

A maioria dos responsáveis pelas famílias concluíram o EM e incentivam os filhos a ingressarem no Ensino Superior, acreditam que a educação é a \emph{garantia de um futuro melhor}.

\subsection{Estatísticas}
Realiza anualmente registros e análises dos principais índices da educação básica e busca implementar ações  de melhoria através de reuniões de estudo e cursos de capacitação de professores. A seguir apresentamos três quadros resultantes destas ações.

\begin{quadro}[ht!]    
    \caption{Taxa percentual de matrículas anual por modalidade de ensino ofertada.}
    \label{qua:totalMatricula}
    \begin{tabular}{|cccccc|}
        \hline
        \multicolumn{6}{|c|}{\textbf{TOTAL DE ALUNOS MATRICULADOS}}                                                                                    \\ \hline
        \multicolumn{1}{|c|}{ANO} &
        \multicolumn{1}{c|}{TOTAL GERAL} &
        \multicolumn{1}{c|}{ENS. FUND} &
        \multicolumn{1}{c|}{\%} &
        \multicolumn{1}{c|}{ENS. MÉD} &
        \% \\ \hline
        \multicolumn{1}{|c|}{2016} & \multicolumn{1}{c|}{610} & \multicolumn{1}{c|}{249} & \multicolumn{1}{c|}{40,8} & \multicolumn{1}{c|}{361} & 59,1 \\ \hline
        \multicolumn{1}{|c|}{2017} & \multicolumn{1}{c|}{601} & \multicolumn{1}{c|}{258} & \multicolumn{1}{c|}{42,9} & \multicolumn{1}{c|}{343} & 57,0 \\ \hline
        \multicolumn{1}{|c|}{2018} & \multicolumn{1}{c|}{663} & \multicolumn{1}{c|}{265} & \multicolumn{1}{c|}{39,9} & \multicolumn{1}{c|}{398} & 60,0 \\ \hline
        \multicolumn{1}{|c|}{2019} & \multicolumn{1}{c|}{672} & \multicolumn{1}{c|}{278} & \multicolumn{1}{c|}{41,3} & \multicolumn{1}{c|}{394} & 58,3 \\ \hline
        \multicolumn{1}{|c|}{2020} & \multicolumn{1}{c|}{700} & \multicolumn{1}{c|}{286} & \multicolumn{1}{c|}{40,8} & \multicolumn{1}{c|}{413} & 59,0 \\ \hline
        \multicolumn{1}{|c|}{2021} & \multicolumn{1}{c|}{741} & \multicolumn{1}{c|}{323} & \multicolumn{1}{c|}{43,6} & \multicolumn{1}{c|}{418} & 56,4 \\ \hline
    \end{tabular}
\end{quadro}

\begin{quadro}[ht!]
    \centering
    \caption{Percentual de desistência/transferência por ano. }
    \label{qua:totalDesistencia}
    \begin{tabular}{|ccccccc|}
        \hline
        \multicolumn{7}{|c|}{\textbf{TOTAL DE ALUNOS DESISTENTES E TRANSFERIDOS}} \\ \hline
        \multicolumn{1}{|c|}{\multirow{2}{*}{ANO}} &
        \multicolumn{1}{c|}{\multirow{2}{*}{\begin{tabular}[c]{@{}c@{}}TOTAL\\ FINAL\end{tabular}}} &
        \multicolumn{2}{c|}{DESISTENTES} &
        \multicolumn{2}{c|}{TRANSFERIDOS} &
        \multirow{2}{*}
        {\begin{tabular}[c]
            {@{}c@{}}TOTAL GERAL DA\\ MOVIMENTAÇÃO
        \end{tabular}}
        \\ \cline{3-6}
        \multicolumn{1}{|c|}{} &
        \multicolumn{1}{c|}{} &
        \multicolumn{1}{c|}{Total} &
        \multicolumn{1}{c|}{\%} &
        \multicolumn{1}{c|}{Total} &
        \multicolumn{1}{c|}{\%} &
        \\ \hline
        \multicolumn{1}{|c|}{2016} &
        \multicolumn{1}{c|}{610} &
        \multicolumn{1}{c|}{12} &
        \multicolumn{1}{c|}{1,96} &
        \multicolumn{1}{c|}{79} &
        \multicolumn{1}{c|}{12,9} &
        701 \\ \hline
        \multicolumn{1}{|c|}{2017} &
        \multicolumn{1}{c|}{601} &
        \multicolumn{1}{c|}{21} &
        \multicolumn{1}{c|}{3,49} &
        \multicolumn{1}{c|}{97} &
        \multicolumn{1}{c|}{16,1} &
        719 \\ \hline
        \multicolumn{1}{|c|}{2018} &
        \multicolumn{1}{c|}{651} &
        \multicolumn{1}{c|}{30} &
        \multicolumn{1}{c|}{4,60} &
        \multicolumn{1}{c|}{102} &
        \multicolumn{1}{c|}{15,6} &
        783 \\ \hline
        \multicolumn{1}{|c|}{2019} &
        \multicolumn{1}{c|}{627} &
        \multicolumn{1}{c|}{12} &
        \multicolumn{1}{c|}{1,91} &
        \multicolumn{1}{c|}{109} &
        \multicolumn{1}{c|}{17,4} &
        748 \\ \hline
        \multicolumn{1}{|c|}{2020} &
        \multicolumn{1}{c|}{690} &
        \multicolumn{1}{c|}{16} &
        \multicolumn{1}{c|}{2,31} &
        \multicolumn{1}{c|}{142} &
        \multicolumn{1}{c|}{20,6} &
        848 \\ \hline
        \multicolumn{1}{|l|}{2021} &
        \multicolumn{1}{c|}{--} &
        \multicolumn{1}{c|}{--} &
        \multicolumn{1}{c|}{--} &
        \multicolumn{1}{c|}{--} &
        \multicolumn{1}{c|}{--} &
        -- \\ \hline
    \end{tabular}
\end{quadro}


\begin{quadro}[ht!]
    \centering
    \caption{Desempenho anual.}
    \label{qua:totalReprovacao}
    \begin{tabular}{|cccccc|}
        \hline
        \multicolumn{6}{|c|}{\textbf{TOTAL DE ALUNOS REPROVADOS}}                                                                                     \\ \hline
        \multicolumn{1}{|c|}{ANO} & \multicolumn{1}{c|}{TOTAL GERAL} & \multicolumn{1}{c|}{APROVADOS} & \multicolumn{1}{c|}{\%} & \multicolumn{1}{c|}{REPROVADOS} & \% \\ \hline
        \multicolumn{1}{|c|}{2016} & \multicolumn{1}{c|}{610} & \multicolumn{1}{c|}{561} & \multicolumn{1}{c|}{91,9} & \multicolumn{1}{c|}{33} & 5,40 \\ \hline
        \multicolumn{1}{|c|}{2017} & \multicolumn{1}{c|}{601} & \multicolumn{1}{c|}{556} & \multicolumn{1}{c|}{92,5} & \multicolumn{1}{c|}{45} & 7,48 \\ \hline
        \multicolumn{1}{|c|}{2018} & \multicolumn{1}{c|}{663} & \multicolumn{1}{c|}{590} & \multicolumn{1}{c|}{90,6} & \multicolumn{1}{c|}{61} & 9,37 \\ \hline
        \multicolumn{1}{|c|}{2019} & \multicolumn{1}{c|}{672} & \multicolumn{1}{c|}{577} & \multicolumn{1}{c|}{92,0} & \multicolumn{1}{c|}{50} & 7,87 \\ \hline
        \multicolumn{1}{|c|}{2020} & \multicolumn{1}{c|}{700} & \multicolumn{1}{c|}{681} & \multicolumn{1}{c|}{98,7} & \multicolumn{1}{c|}{9}  & 1,31 \\ \hline
        \multicolumn{1}{|c|}{2021} & \multicolumn{1}{c|}{--}  & \multicolumn{1}{c|}{--}  & \multicolumn{1}{c|}{--}   & \multicolumn{1}{c|}{--} & --   \\ \hline
    \end{tabular}
\end{quadro}
\newpage
\section{Projeto Político Pedagógico}
Em \cite{CARVALHOAMP:2012a}, o Projeto Político Pedagógico (PPP) é um documento representativo do pensamento escolar, destaca a importância da elaboração deste documento em conjunto com a comunidade; pais, professores, representantes de alunos e equipe diretiva. Uma atenção ainda é dada a necessidade do documento manter-se disponível à todos os entes desta comunidade. 

Um outro trabalho desenvolvido por \cite{VEIGA:1995}, vai de encontro a esta visão e o traz como elemento resultante de uma ação intencional de sentido explícito e compromisso definido coletivamente.
\begin{citacao}
    ``\ldots todo projeto pedagógico de uma escola é, também, um projeto político por estar intimamente articulado ao compromisso sociopolítico com os interesses reais e coletivos da população majoritária. É político no sentido de compromisso com a formação do cidadão para um tipo de sociedade. A dimensão política se cumpre na medida em que ela se realiza como prática especificamente pedagógica. Na dimensão pedagógica reside a possibilidade da efetivação da intencionalidade da escola, que é a formação do cidadão participativo, responsável, compromissado, crítico e criativo. Pedagógico, no sentido de definir as ações educativas e as características necessárias às escolas de cumprirem seus propósitos e sua intencionalidade.'' \opcit[p. ~2]{VEIGA:1995}
\end{citacao}
Dessa forma entende-se que tanto o processo de construção quanto o a aprovação do PPP deva seguir tais preceitos. Cabe ainda à gestão escolar providenciar formas de viabilizar a participação de todos os membros da comunidade escolar. 

\subsection{Atribuição dos agentes}
A GPF posiciona-se de acordo com estes referenciais ao prever em seu PPP
\begin{citacao}
    ``\ldots A participação dos professores e especialistas na elaboração do projeto pedagógico promove uma dimensão democrática na escola e nessa perspectiva, as decisões não centralizadas no Gestor cedem lugar a um processo de fortalecimento da função social e dialética da escola por meio de um trabalho coletivo entre todos os segmentos participantes e a comunidade escolar. '' \cite[p. ~5]{GPTPPP:2021}''
\end{citacao}
dispõe do \emph{Conselho Escolar} e do \emph{Conselho de Classe}, como instâncias criadas para garantir a representatividade, legitimidade e continuidade das ações educativas. Há somente uma cópia do PPP que fica na sala da ATP e pode ser consultado por qualquer pessoa que se interessar.

\subsection{Abordagem curricular}
A abordagem curricular para o EM, tem por objetivo geral, proporcionar ao aluno rigor conceitual, conhecimento sistematizado, organização de estudos e confiança nos resultados como forma de melhorar sua autoestima, responsabilidade e preparação para a vida prática. Está alinhada à BNCC, no que concerne alguns de seus objetivos
\begin{citacao}
    ``\ldots desenvolver nos alunos habilidades e competências que serão o suporte para criação em áreas diversas e para a resolução de situações-problemas pessoais ou coletivos ao o longo de sua vida'' \cite[p. ~32]{GPTPPP:2021}
\end{citacao}
Já por objetivos específicos, visa a aplicação da autonomia e da cidadania, do senso crítico e da criatividade, tanto nas rotinas escolares quanto nas atividades extracurriculares, dentre outros.

Para atingir aos objetivos mais específicos, a GPF oferece matrizes curriculares no EM de acordo com as normativas do Estado, tem por metodologia, promover o protagonismo do aluno, favorecendo a estruturação e expansão do conhecimento, neste sentido, menciona a mediação como ação principal do professor, tendo por uma de suas competências, compreender como o aluno constrói o conhecimento para que a aprendizagem se consolide de forma significativa.

\subsection{Avaliação}
O processo de avaliação na visão da GPF é entendida como um processo pelo qual deve adequar-se à natureza da aprendizagem, levando em consideração não os fins mas sim a trajetória do aluno no decorrer do processo formativo. Os resultados das avaliações devem servir também como prática reflexiva do professor e, quando necessário, para o redirecionamento do processo de ensino-aprendizagem, além de um importante instrumento que possibilite ao aluno tomar consciência não só de suas dificuldades como também de seus avanços e potencialidades. O documento ainda reserva ao professor a abertura de empregar diferentes estratégias de avaliação, sugerindo além de provas, trabalhos em grupos, exercícios de fixação, apresentações orais e escritas, dentre outros. A recuperação paralela é prevista, devendo ser ministrada continuamente por meio de correções de deveres e exercícios, após cada avaliação quando necessário, após cada unidade trabalhada, retomando as atividades e incentivando o aluno à prática da autocorreção.
