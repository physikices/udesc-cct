\chapter{Apresentação da Concedente}
\label{cap: aprConcedente}

Um estudo elaborado por \cite{CARVALHOAMP:2012a} recomenda aos estagiários a tomarem conhecimento do ambiente escolar primeiramente por intermédio da observação. Buscando entre uma observação e outra compreender a realidade a que estará inserido no decorrer de sua atuação profissional, em tese isso deve contribuir para orientar a prática pedagógica do futuro professor e fornecer-lhes dados subsidários, para uma formação sólida e humanizada.
\begin{citacao}
``O conhecimento de uma escola passa pelo conhecimento de sua parte física e humana.(...)mas principalmente com a sua sensibilidade (do estagiário). Ela é limpa? É um ambiente agradável fisicamente? Um membro da equipe diretora está sempre presente na escola?'' \Ibidem[p. ~5]{CARVALHOAMP:2012a}
\end{citacao}
Adotando esta recomendação, dar-se-á por iniciada nossas atividades.

De muros altos e pintados de verde a Giovani Pasqualini Faraco se mistura à paisagem urbana da região norte de Joinville, não passa por despercebida pois tudo em sua estrutura, ainda que de certa forma, escondida pelos altos muros; denuncia o padrão estrutural das unidades escolares públicas da região. Dessa forma os muros não apenas lhe conferem \emph{proteção}, mas também referência.

O acompanhamento dos muros nos leva a entrada principal, consiste num portão largo, aberto ao início e final de cada turno, é por onde passam a maioria dos alunos e dos atuomóveis, um segundo portão feito de grade e estreito é visto mais adiante, este da acesso à secretaria.

\section{Caracterização da Unidade Escolar}
De clima agradável, muito bem equipada e limpa a secretaria é a primeira instância administrativa de que a comunidade tem acesso direto à escola, há sempre alguém de prontidão e não demora muito a atendê-lo. Folhetos no balcão e cartazes colados nos murais, informam as datas e atividades que a escola desenvolve em conjunto com a comunidade. Um caminho estreito e florido, liga a secretaria ao rol onde ficam; as salas dos(as) Assistentes Técnicos(os) Pedagógicas(os), da Direção e a dos Professores.

\subsection{Infraestrutura}
Tem por área construída 3114,50 $m^2$ e outros 13929,25 $m^2$ de área disponível. Conta com uma infraestrutura de; 17 salas de aulas  de 42 $m^2$ cada, cozinha, depósito para merenda escolar, depósito para produtos de limpeza, cantina, banheiros feminino e masculino, banheiros para professores, biblioteca, laboratório para as disciplinas de Física/Química e Biologia, laboratório de informática, sala ambiente para a disciplina de língua Portuguesa, arquivo inativo, sala para materiais de Educação Física, sala para materiais e equipamentos de: orientação escolar, direção, vídeo, artes e materiais para docentes.

A área descoberta é bem agradável e limpa, possui árvores ao seu redor criando um ambiente convidativo para o exercício da leitura, recreação ou até mesmo aulas diversificadas. Possui também uma pérgola onde realizam aulas de leitura. Para um contato maior com a terra, dispõe de uma horta escolar, onde os alunos podem fazer o reconhecimento de hortaliças e vegetais.

Possui pátio e quadras cobertas, onde são feitas as atividades culturais e esportivas, além das refeições no horário do recreio.

\subsection{Recursos Humanos}


\section{Histórico}
Fundada em 15 de fevereiro de 1938, a Escola Desdobrada Dona Francisca Quilômetro Cinco, fica situada junto à Rua Dona Francisca, número 4957, Bairro Santo Antônio -  Joinville/SC. Têm em seu nome o Patrono da instituição, Giovaini Faraco.

Nascido em 05 de abril de 1915, filho de Biase Faraco e Maria Limongi Faraco, conclui o Ensino Primário na Escola São José, ingressando em seguida no Ginásio onde permanece como aluno interno até o ano de 1929. Em 1930 entra pro Seminário Central de São Leopoldo, onde permanece até o ano de 1935. Ao passo de sua jornada, lecionou latim e português em diversos estabelecimentos particulares. Possui boa desenvoltura musical e compôs várias partituras para o instrumento órgão. Falece em 10 de novembro de 1960, vítima de hipertensão arterial, deixando sua esposa Tusnelda Gomes Faraco e cinco filhos.

O decreto de n$^\circ$ 10138/70 renomeia a unidade para Grupo Escolar Giovani Pasqualini Faraco e em 30 de junho de 2004, implanto o Ensino Médio, passando a integrar a rede Estadual de Ensino.

\section{Projeto Político Pedagógico}

\subsection{Concepções Norteadoras}
O Projeto Pedagógico Institucional do IFSC toma como ponto de partida o marco referencial teórico-metodológico elaborado e construído de forma coletiva pelos integrantes da comunidade escolar. As concepções norteadoras explicitadas neste documento os fundamentos básicos que orientarão a formulação de diretrizes, políticas e projetos da instituição, e atuarão como bases da unidade do IFSC em seu processo de planejamento, execução e avaliação dos planos de ensino, pesquisa e extensão.
\subsection{Apresentação do PPI}
O Projeto Pedagógico Institucional do IFSC toma como ponto de partida o marco referencial teórico-metodológico elaborado e construído de forma coletiva pelos integrantes da comunidade escolar. As concepções norteadoras explicitadas neste documento constituirão os fundamentos básicos que orientarão a formulação de diretrizes, políticas e projetos da instituição, e atuarão como bases da unidade do IFSC em seu processo de planejamento, execução e avaliação dos planos de ensino, pesquisa e extensão. \cite{GPTPPP:2021}
\subsection{Sistema de Avaliação}