
%============| INÍCIO |=======================================>
\chapter{Introdução}
\lettrine{D}{esde} o seu surgimento no século XI o termo estágio \emph{(do lat.: stagium)} tem sido associado à aprendizagem posta em prática num local adequado e sob supervisão. Durante a trajetória acadêmica, os saberes adquirido ao longo dos anos de formação são postos em prática buscando estabelecer vínculos entre o saber e o saber fazer, acompanhado por um profissional da área que orientará e corrigirá as ações desenvolvidas pelo estagiário e assim, evitar falhas no exercício de suas atribuições no momento em que estiver apto a desenvolvê-las.

Neste sentido os estágios representam para o estagiário uma oportunidade de colocar em prática os conhecimentos construídos pelo futuro profissional ao longo de todo o processo formativo, esta prática é tanto mais proveitosa quando porporcionada à situações concretas e próximas da realidade profissional. Embora nestes termos não esteja explicitado, vale ressaltar que a prática do estágio jamais deve ser confundida com aplicação de mão de obra barata, a Lei de n$^\circ$ 11.788, de 25 de setembro de 2008 determina que o estágio seja vinculado puramente ao processo educativo. O conhecimento da normal e a gestão correta do estágio pelas instituições devem ser suficientes para evitar que está pratica se difunda na forma de precarização das relações do trabalho.

\section{Documentos Norteadores da Educação Básica}
Com a homologação da 3ª versão da Base Nacional Comum Curricular (BNCC) em dezembro de 2017, passa então a valer em todo o território nacional, em caráter compulsório e de forma prevista pela Lei de Diretrizes de Bases da Educação Nacional (LDB) assim como no Plano Nacional de Educação (PNE), as políticas educacionais voltadas a orientar a elaboração: dos currículos locais, da formação inicial e continuada dos professores, do material didático, da avaliação e do apoio pedagógico aos alunos \cite{BRASIL:2017}, a fim de assegurar e promover os direitos de aprendizagem essenciais aos educandos com vistas à formação humana integral e à construção de uma sociedade justa, democrática e inclusiva.

O texto tem como foco o desenvolvimento de \emph{competências} por meio das quais o educando, ao longo de todo o processo formativo, deva ser capaz de
\begin{citacao}
    ``\ldots aprender a aprender, saber lidar com a informação cada vez mais disponível, atuar com discernimento e responsabilidade nos contextos das culturas digitais, aplicar conhecimentos para resolver problemas, ter autonomia para tomar decisões, ser proativo para identificar os dados de uma situação e buscar soluções, conviver e aprender com as diferenças e as diversidades.'' 
    \Ibidem[p. ~14]{BRASIL:2017}   
\end{citacao}

Face a isso, os cursos de licenciaturas do país tem buscado promover nos currículos de graduação, o conjunto de ações adequadas a atender às exigências dos documentos norteadores. Neste processo, encontram-se as disciplinas de Estágio Curricular Supervisionado I/II/III e IV, responsáveis por oportunizar uma primeira aproximação do acadêmico com a carreira docente em ambiente escolar supervisionado, sendo um componente curricular obrigatório e indispensável nos cursos de licenciatura, de acordo com a resolução \cite{BRASIL:2002a} homologada pelo Conselho Nacional de Educação (CNE) na forma do parecer de nº  CNE/CP nº 1, de 18 de Fevereiro de 2002.

\section{Referênciais Teórico-Metodológico}
Longe de reduzir a ação dos docentes como meros agentes tecnicistas, limitados a cumprir passivamente o que lhes ditam verticalmente, vê-se nos textos da BNCC, uma proximidade com as concepções da filosofia \emph{deweyana}\footnote{John Dewey (1859-1952), filósofo americano que influenciou educadores de várias partes do mundo e que no Brasil inspirou o \emph{Movimento da Escola Nova}, liderado por Anísio Teixeira.}, neste sentido, considera-se o movimento da \emph{Prática Reflexiva} proposta por \cite{ZEICHNER:1993} como elemento catalizador do pensar e repensar frequentemente a prática pedagógica, nesta na visão
\begin{citacao}
``O conceito de professor como prático reflexivo reconhece a riqueza da experiência que reside na prática dos bons professores. Na perspectiva de cada professor, significa que o processo de compreensão e melhoria do seu ensino deve começar pela reflexão sobre a sua própria experiência e que o tipo de saber inteiramente tirado da experiência dos outros (mesmo de outros professores) é, no melhor dos casos, pobre e, no pior, ilusão.'' \opcit[p. ~17]{ZEICHNER:1993}
\end{citacao}
Não se trata aqui de tornar o estagiário durante o exercício do estágio, um crítico contumaz à pratica docente observada em sala de aula, mas sim de fazê-lo 
\begin{citacao}
``[...]detectar e superar uma visão simplista dos problemas de ensino e aprendizagem, proporcionando dados significativos do cotidiano escolar que possibilitem uma \textbf{reflexão crítica} do trabalho a ser desenvolvido como professor e dos processos de ensino e aprendizagem em relação ao seu conteúdo específico.'' \cite[p. 11, \textbf{grifos meus}]{CARVALHOAMP:2012b} 
\end{citacao}
Assim sendo, as problematizações trazidas à tona neste trabalho, só tem sentido se vistas no âmbito de elucidar a complexa relação existente entre o ato de ensinar e a aprendizagem significativa desejada, à luz destes referenciais.

\section{Contexto}
Este estágio foi desenvolvido ao longo do primeiro semestre do ano de 2022, na Escola de Educação Básica (EEB) Giovani Pasqualini Faraco (GPF), para a disciplina de Estágio Curricular Supervisionado II do curso de Licenciatura em Física, onde o estagiário é convidado a desenvolver atividades relacionadas à caracterização do ambiente escolar, acompanhamentos de aulas além de proposição e execução de atividades imersivas.
 
O restante desse trabalho está organizado da seguinte maneira: no Capítulo \ref{cap: aprConcedente} é apresentada a unidade concedente do estágio, suas características estruturais e organizacionais; o Capítulo \ref{cap: apoioDocencia} é destinado à apresentação dos programas de apoio à docência; no Capítulo \ref{cap: prgFisica} faremos a apresentação dos programas da disciplina de Física; no Capítulo \ref{cap: acmpDeAulas} apresentaremos as intervenções feitas em sala de aula; já no Capítulo \ref{cap: reunPedagogica} falaremos sobre o acompanhamento das reuniões pedagógicas e por fim, as considerações finais são apresentados no Capítulo \ref{cap: consideracoesFinais}.




%=============| FIM |=========================================>
	